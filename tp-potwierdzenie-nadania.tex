%& --translate-file=il2-pl
%% 
%% Szablon do sk�adania polecenia nadania (list polecony)
%% Druk jest nieco mniejszy ni� orygina� (usu� \magnification 975 je�eli ma by� identyczny)
%% Dzi�ki zmniejszeniu wymiar�w na jednej stronie A4 zmie�ci si� 4x2 = 8 druczk�w
%% (c) t.przechlewski, 2012, http://www.gnu.org/licenses/gpl-3.0.txt
%% 
\magnification 975
\input tp-boxes.tex
\input pdf-trans.tex %% p. jackowski's pdf trans macros
\def\TPifundefined#1#2#3{\expandafter\ifx\csname #1\endcsname\relax#2\else#3\fi}
%%
\TPifundefined{UserFontHt}{\gdef\UserFontHt{11pt}}{}%%%
%% Wykorzystujemy fonty TeXGyre + Antykw�P�tawskiego
\font\UserFont= qx-antpr10 at \UserFontHt
\font\Title = qx-qhvcr-sc at 12pt
\font\title = qx-qhvcr-sc at 10pt
\font\normal = qx-qhvcr at 8pt
\font\small = qx-qhvcr at 6pt
\font\plsy = plsy10 at15mm
\font\descrfont = qx-qhvcri at 6pt
\normal 
\baselineskip6mm
%%
\def\IMIENAZWISKO#1{\gdef\CImienazwisko{#1}}
\def\ULICA#1{\gdef\CUlica{#1}}
\def\KODPOCZTOWY#1{\gdef\CKodpocztowy{#1}}
\def\MIASTO#1{\gdef\CMiasto{#1}}
%%
\def\NADAWCAIMIENAZWISKO#1{\gdef\Cnadawcaimienazwisko{#1}}
\def\NADAWCAULICA#1{\gdef\Cnadawcaulica{#1}}
\def\NADAWCAKODPOCZTOWY#1{\gdef\Cnadawcakodpocztowy{#1}}
\def\NADAWCAMIASTO#1{\gdef\Cnadawcamiasto{#1}}
%% Odst�p od kropek do linii bazowej tekstu:
\def\AboveDots{2pt}
%% 
%%\def\kolko{\centerline{\plsy\char13}}
%%\def\kolko{\centerline{\boxscale{330}\hbox{$\bigcirc$}}}
\def\kolko{\centerline{\pdfximage width 14mm {kolko14mm-1.pdf}\pdfrefximage \pdflastximage}}
%
\def\OnDots#1{\raise\AboveDots\hbox to0pt{\UserFont\quad #1\hss}\dotfill}
\def\BOX {{\MarginSep = 2.0mm \boxit{\hbox to0pt{\hss}}}}
\def\Line#1{\hbox to \BoxWdLt{#1\dotfill}}
\def\kodpocztowy#1{\hbox to0pt{\raise-2ex\hbox to20mm{\hss\small kod pocztowy\hss}\hss}%
  \hbox to0pt{\UserFont \quad #1\hss}\hbox to20mm{\dotfill}~}
%% %%
\newdimen\BoxWd \BoxWd=95.0mm
\newdimen\BoxWdLt \BoxWdLt=70mm
\newdimen\BoxHd \BoxHd=70mm
\MarginSep = 0.5mm %%% 2.5mm
%%
\newbox\EtykietaBoxL \newbox\EtykietaBoxP
\def\EtykietaL{\PrzygotujEtykiete{\EtykietaBoxL}}
\def\EtykietaP{\PrzygotujEtykiete{\EtykietaBoxP}}
\def\PrintRow{\par %%\vskip0mm\hrule
  \vskip1.5mm \hbox{\box\EtykietaBoxL\kern4mm\box\EtykietaBoxP}}
%
\long\def\PrzygotujEtykiete#1{%%
\setbox #1 \vbox{ %\Lboxit{
\vbox to \BoxHd{\hsize = \BoxWd
%%%
\baselineskip4mm plus 5mm
%%%
\centerline {\Title potwierdzenie nadania}\vskip1mm
\hbox to\BoxWd{\title przesy�ki poleconej nr \dotfill}
%%
%%
\hbox to 100mm{%
\MarginSep = 0.5mm
\vbox{%%
\hbox to\BoxWdLt{\hss\descrfont wype�nia nadawca \hss}
\Lboxit{%%
\vbox to 56mm{\hsize=\BoxWdLt
%\line{\dotfill wype�nia nadawca \dotfill}
\kern5.0mm
\Line{{\title nadawca}:~\dotfill}
\Line{\OnDots{\Cnadawcaimienazwisko}}
\Line{\OnDots{\Cnadawcaulica}}
\Line{\kodpocztowy{\Cnadawcakodpocztowy}\OnDots{\Cnadawcamiasto}}
\vskip2.5mm
\Line{{\title adresat}:~\dotfill}
\Line{\OnDots{\CImienazwisko}}
\Line{\OnDots{\CUlica}}
\Line{\kodpocztowy{\CKodpocztowy}\OnDots{\CMiasto}}
}}}
%%
\vbox to 56mm{\hsize = 27mm
\baselineskip6.6mm plus 6mm
%%
\kern1mm
\line{Op�ata \dotfill z� \dotfill gr}\par
\line{Masa: kg \dotfill  g~\dotfill}\par
\line{Gabaryt A \BOX ~~~B \BOX \hss}\par
\line{Priorytetowa \BOX \hss}\par
\line{Potwierdzenie odbioru \BOX \hss}
%%
\vskip2.5mm
%%
\kolko 
\vskip-3.5mm
\centerline{Podpis przyjmuj�cego}

}}}}}
%%
%% Dopasowanie margines�w strony A4
\advance \hoffset-22mm
\advance \voffset-24mm
\advance \vsize 60mm
\nopagenumbers
\endinput
%%
%%
%% Przyk�ad wykorzystania (wymagana jest kompilacja pdftexem, nie texem ani nie pdf/latexem)
%% ----------------------
%%\def\UserFontHt{10pt} %% <-- domy�lnie 11pt
%%\input tp-potwierdzenie-nadania.tex
%%
%%\NADAWCAIMIENAZWISKO{ks. pra�at Wincenty Jacenty Zadro�ny}
%%\NADAWCAULICA{Aleja Zwyci�stwa 128a/217}
%%\NADAWCAKODPOCZTOWY{81-825}
%%\NADAWCAMIASTO{Brze�� Kujawsko-Pomorski}
%%
%%\IMIENAZWISKO{Jan Kowalski}
%%\ULICA{Abrahama 28a/17}
%%\KODPOCZTOWY{81-825}
%%\MIASTO{Sopot}
%%\EtykietaL
%%
%%\IMIENAZWISKO{Kowalska Janina}
%%\ULICA{Abrahama 28a/17}
%%\KODPOCZTOWY{81-825}
%%\MIASTO{Sopot}
%%\EtykietaP
%%
%%\PrintRow 
%%